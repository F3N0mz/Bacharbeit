\section{Results Analysis and Discussion}
The project was very extensive and diverse in types of tasks one person had to do.I would consider the end result to be rather \textbf{Unsuccessful}. In this chapter I will reflect and discuss on the process and provide my \textbf{personal} feedback on what went well, could've been done better, and what would be needed to be done next. It is worth going through the problems first, so that it provides the context for the final result of the project.
\subsection{Problems}
The problems that have occurred can be divided into 2 big categories. First is the problems of \textbf{Organization and Project Management} These are the issues of the structure. Unfortunately, I have never had any experience before in managing a successful \textbf{Project}. Although I have work experience, my tasks were primarily immanent ones, those that either don't last long or don't require deep planning. It might have actually negatively affected my outcome for this project as I worked under a rather arrogant assumption that my experience would be of use here also, but it was actually the opposite. My work experience made me somewhat negligent of the due process of dividing this (quite extensive) task into the smaller pieces and tackle them one by one. This is what I would call structure and this my project lacked. 

The other problem with organization of the project is ambiguity of the roles that the participants would play. I came to the project management with somewhat \textbf{Agile} paradigm in mind. Which meant that there would be a second player from whom I would receive constant feedback, but the issue was that while there was an agile mindset, there was no agile project management and there was also no such person that would provide actionable, useful feedback. This state hindered the speed of development in an already quite extensive project somewhat. Adding to that, because there was no clear division of tasks and this was what felt like to be my personal project, receiving whatever feedback I would get felt awkward and hard to act upon, even if that feedback was good. Another problem of doing everything personally was the issue of \textbf{divided attention} where everything would have to be kept in mind by one person and jumping from one task to the other felt very disorienting. On the other hand, this experience has shed the light on importance of proper documentation of different areas of development in the project: it allows different teams/departments/organizations to communicate better with each other by laying down definitions and overview of the structure.

Project management
	Task was too big
	Everything done by one person (with the assumption that there would be team work)	
Personal
	Reluctance to engage deeper
	Lack of motivation
	general state of disorganization
