\section{Results Analysis and Discussion}\label{sec:discussion}

%---CITE?
%For with much wisdom comes much sorrow, and as knowledge grows, grief increases.
%Ecclesiastes 1:18 
This section provides an analysis of the project's outcomes and a critical discussion of the development process. The project involved a wide range of tasks spanning multiple technical domains. While valuable experience was gained and certain components were progressed, such as 3D printed body, an existing prototype of an Android app and some backend code, the overall integration and functionality targeted by the initial objectives were not fully achieved. Consequently, the final result is evaluated as not meeting the predefined success criteria. The success criterion was this: A fully functioning Pill Dispenser with a remote app to remotely control it. This discussion will explore the contributing factors, focusing on challenges encountered in project management and technical execution, reflect on the lessons learned, and propose directions for future work. Contextualizing these challenges is crucial for understanding the project's trajectory and final state.

\subsection{Challenges Encountered}

The difficulties faced during the project can be broadly categorized into two interconnected areas: \textbf{Project Management and Execution}, and \textbf{Technical Complexity and Resource Constraints}.

\subsubsection{Project Management and Execution}

Effective project management proved to be a significant challenge, stemming primarily from a lack of prior experience in managing long-term, multi-faceted technical projects.

\begin{enumerate}
	\item \textbf{Planning and Structuring} Early momentum was shaped by \textit{over-confidence that past short, ad-hoc jobs had prepared the author for end-to-end project ownership}. . An initial underestimation of the need for rigorous, upfront planning led to difficulties. Specifically, the project lacked a sufficiently detailed breakdown of the overall task into manageable sub-components with clear milestones and dependencies. This absence of a granular structure hindered effective progress tracking and adaptation throughout the development cycle.
	\item \textbf{Resource Availability} Ambiguity regarding the scope of permissible resources (e.g., access to university facilities, specific software licenses, or hardware) and their integration into the project plan was an early challenge. This lack of explicit definition led to a predominantly Do-It-Yourself approach to problem-solving, potentially precluding the use of more optimized or readily available solutions typical in industrial or well-resourced academic environments. This highlighted the necessity for proactive clarification of all available resources at the project's outset.
	\item \textbf{Development Methodology and Feedback Loop}: An Agile-inspired iterative approach was initially assumed, anticipating regular feedback to guide development. However, the conditions \textbf{were not clearly defined} and there was some confusion about what role each participant (including me) would play. There was an absence of a clearly defined counterpart or stakeholder role dedicated to providing consistent, actionable feedback. This ambiguity made it challenging to effectively solicit, interpret, and integrate feedback, occasionally impeding development velocity. Furthermore, the lack of clearly delineated responsibilities within the project (as it was executed solely by the author) made whatever feedback was given feel less integrated into a structured workflow. 
	\item \textbf{Task Management and Cognitive Load}: Executing all project tasks single-handedly – from conceptualization and design to implementation and testing across different domains – resulted in significant challenges related to \textbf{divided attention}. Constantly shifting between diverse tasks (e.g., 3D modeling, embedded programming, mobile development) proved cognitively demanding and potentially reduced efficiency and focus in each area. This experience highlighted a key learning: the critical importance of comprehensive documentation. Well-structured documentation serves as an essential tool for defining components, standardizing terminology, and maintaining a clear overview, thereby facilitating better intrapersonal communication (tracking one's own work across domains) and enabling potential future collaboration.
\end{enumerate}

\subsubsection{Technical Complexity and Resource Constraints}

The project's ambitious scope required integrating expertise from several distinct technical fields, which posed considerable problems for a single developer.
\begin{enumerate}
	\item \textbf{Scope and Expertise Mismatch}: The project necessitated skills in 3D modeling and printing, Android application development, and microcontroller programming with Bluetooth Low Energy (BLE) communication. While 3D design and printing were navigated successfully, providing valuable practical insights, they consumed a significant portion of the available time.
	\item \textbf{Software Development Learning Curve}: The software development aspects presented steep learning curves:
	\begin{enumerate}
		\item Android Development: Required learning not only Kotlin syntax but also the \ac{OOP} paradigm as applied to GUI development, Android-specific \ac{API}, the application lifecycle, build toolchains (e.g., Gradle), and handling platform diversity (\ac{API} levels, device compatibility). This differs substantially from procedural or scripting languages like C, Python, or MATLAB, particularly concerning memory management, data type handling, and runtime environments.
		
		\item Microcontroller Programming \& \ac{BLE}: Required dedicated research into embedded C/C++, microcontroller architecture (referencing datasheets), and the complexities of the BLE protocol stack and its implementation on the target hardware. While manufacturer documentation and libraries provide support, integrating these components effectively remained time-intensive.
		
		\item Effort Underestimation: The cumulative effort required to gain proficiency and implement solutions across these diverse domains was initially underestimated. The need to simultaneously learn and apply knowledge in unfamiliar areas significantly impacted the development timeline and the ability to achieve the desired level of integration and polish for the final system. Tackling these complex, disparate technical challenges single-handedly proved to be a major constraint.
	\end{enumerate}
\end{enumerate}
\newpage
\subsection{Reflection and Lessons Learned} 
Despite the project not fully meeting its objectives, the process yielded significant learning outcomes. The development itself delivered a big amount of various domain knowledge in areas of 3D-Design and Printing, Android and Microcontroller development and Project Management:
\begin{enumerate}
	\item \textbf{3D-Design and Printing}: Before starting the project, my experience designing a functional device was limited. Moreover, the designed components would then have to be 3D-printed, which meant that material properties of the filament would have to be taken into account, leading to research and adjustment of the design to facilitate the newly obtained knowledge.
	\item \textbf{Android Development}: Through this project a lot of insights has been gained on how the Android development environment works. Since I had no experience in programming any User interface harder than basic Command-Line interface, I had to learn how to do it and also using \ac{OOP} for that. Android platform has also added complexity, since one would need to learn how Application Packaging, Permissions, Device Compatibility and many other components interact together in an Android Application.
	\item \textbf{Microcontroller Development} I have already had experience developing on an another microcontroller, which was also rather lower-level. Arduino and Platform.io development environments have made it easier to write a backend, however writing a \ac{BLE} was new for me. 
	\item \textbf{Project Management} This project has had a profound impact on my view of the role and importance of Project Management. It is important to not also divide the tasks in a project-stucture plan (which has-been done), but also to clarify the participants, their roles and tasks. Since this hasn't been done initially it lead to difficulties estimating necessary amount of effort to complete the project successfully. It is also important to know in advance (and have it written down somewhere) which resources are made available for the project. There has been no requirements on budget laid down. Generalizing, the high amount of requirements is good because it reduces amount of creative freedom. There is a balancing act at play where if the requirements are too strict, there is only one solution or none at all, while on the other hand, if the requirements are too lax, the creative freedom is too wide which leads to kind of analysis paralysis, in which the developer would have to either lay down requirements themselves (effectively delegating the task of creating a Technical Task onto the developer) or operate on assumptions (which would then require a feedback from the client, whether these assumptions are adequate).
\end{enumerate}


\subsection{Next steps}
A lot of work remains to be done. From the side of components it might be worth looking into a stronger stepper motor, as the current one has very narrow margin of error and its performance is highly dependent on battery level. What this means is that when the battery level is low, it might not be able to pull the load of 2 mills and extra weight. The construction of the device itself is mostly fine, although the chamber wheels are a bit loose to reduce amount of contact area and friction. If there is a change in components, it would most certainly also require redisigning the lower deck to fit the new components, as the more powerful stepper motor would also require different (perhaps also bigger) driver.  Replacing the motor would also require better power supply, as a single 18650 battery might not be enough to power it. 

Another area of improvement would be the addition of sensors. Adding sensors would allow more precise control of the chambers (e.g. using hall sensors + magnet on each wing to check whether chambers are aligned), as well as tracking whether the pills were dispensed or disposed using an accelerometer at the bottom of dispense chamber.
The reason it hasn't been initially done is the lack of documented requirements. On one hand, the project is already quite extensive as it is, on another hand, implementing any of the above mentioned changes would require allocating resources that have already been critically lacking. Doing this would mean revamping a substantial part of the device construction, therefore it is better left for the next iteration of development.

There is also a big area for improvement in the Android App. Due to limited experience and knowledge about current state-of-the-art frontend development paradigm, its usability could further be improved. For this, information like user journey(what functions does user use the most and how long does it take them to get there) need to be collected. Certain functions could also be implemented the other way. For example, bluetooth connection could be done through Android settings and device pairing, removing the need for the device connection screen completely.




%Project management
%	Task was too big
%	Everything done by one person (with the assumption that there would be team work)	
%Personal
%	Reluctance to engage deeper
%	Lack of motivation
%	general state of disorganization
%