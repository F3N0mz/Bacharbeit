\section*{Executive Summary}
The end result of the project is a working prototype of the device. The specifications of the device were defined at the beginning of the project and are as follows:
\begin{enumerate}
\item{\textbf{The device contains 21 chambers, 3 for each day of the week.}} 

This requirement is a necessary condition for every pill dispenser, not just the automatic ones, because even the simple dispensers (normally) have a separate chamber for each day.
\item{\textbf{The device contains a pill disposal system.}}

 This is a place where unused pills are stored. This location is also divided into sections so that the caregiver is able to investigate which pills have not been taken and when. Meeting this requirement makes it easier for the caregiver to manage the device, as it is easy for them to analyze unused pills and dispose of the remaining pills.
 
\item{\textbf{The device has an accompanying app for remote control.}} 

This requirement enables the caregiver to control the device remotely. This is useful for the following reasons:

\begin{enumerate}
	\item The device lacks a control panel on the housing. Therefore, the function cannot be interrupted by accidentally pressing a button.
	\item With a single Android app, the caregiver may be able to connect to and manage multiple devices remotely. This is useful in hospitals or retirement homes, for example, where there may be several such devices in a confined space.
	\item Alternatively, for use at home, multiple apps can be connected to a single device (not simultaneously) if there are multiple independent caregivers for a single device.
\end{enumerate}
\end{enumerate}

