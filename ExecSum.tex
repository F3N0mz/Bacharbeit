\section*{Executive Summary}
Das Endergebnis des Projekts ist ein funktionierender Prototyp des Geräts. Die Spezifikationen des Geräts wurden zu Beginn des Projekts festgelegt und lauten wie folgt:
\begin{enumerate}
\item{\textbf{Das Gerät enthält 21 Kammern, 3 für jeden Tag der Woche.}} 

Diese Anforderung ist die notwendige Bedingung für jeden Pillenspender, nicht nur für die automatischen, denn auch die einfachen Spender haben (normalerweise) für jeden Tag eine eigene Kammer.
\item{\textbf{Das Gerät enthält ein Pillenrücknahmesystem }}

 Das ist ein Ort, an dem unbenutzte Pillen aufbewahrt werden. Dieser Ort ist auch in Abschnitte unterteilt, damit der Betreuer des Geräts in der Lage ist, zu untersuchen, welche Pillen wann nicht eingenommen wurden. Die Erfüllung dieser Anforderung erleichtert dem Betreuer die Verwaltung des Geräts, da es für ihn einfach ist, nicht eingenommene Pillen zu analysieren und die restlichen Pillen zu entsorgen.
 
\item{\textbf{Das Gerät verfügt über eine begleitende App zur Fernsteuerung.}} 

Diese Anforderung ermöglicht es dem Hausmeister, das Gerät aus der Ferne zu steuern. Dies ist aus den folgenden Gründen nützlich:

\begin{enumerate}
	\item Dem Gerät fehlt ein Bedienfeld am Gehäuse. Daher kann die Funktion nicht durch versehentliches Drücken einer Taste unterbrochen werden.
	\item Mit einer einzigen Android-App kann der Betreuer möglicherweise aus der Ferne eine Verbindung zu mehreren Geräten herstellen und diese verwalten. Dies ist z. B. in Krankenhäusern oder Altenheimen nützlich, wo es auf engem Raum mehrere solcher Geräte geben kann.
	\item Für die Nutzung zu Hause können alternativ mehrere Apps mit einem einzigen Gerät verbunden werden (nicht gleichzeitig), wenn es mehrere unabhängige Betreuer für ein einziges Gerät gibt.
\end{enumerate}
\end{enumerate}

